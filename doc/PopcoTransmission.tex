\documentclass[12pt]{article}
\usepackage{natbib}
\usepackage{xspace}
\usepackage{graphicx}
%\usepackage{url}
\usepackage{relsize} % provides \relscale, \larger, etc. to be used *in the document*
\usepackage{color}
\XeTeXdashbreakstate=1 % Tell XeLaTeX it's OK to break lines after em-dashes.
\newcommand{\SetFont}[1]{\setmainfont[Ligatures=TeX]{#1}} % preferred to "Mapping=tex-text"
\newcommand{\UseGaramond}[1]{%
	\usepackage[urw-garamond]{mathdesign}  % goes before fontenc, fontspec, etc.
	\usepackage[T1]{fontenc}
	\usepackage{fontspec,xltxtra,xunicode}
	\SetFont{#1}}
%\UseGaramond{EB Garamond}\newcommand{\myparagraph}[1]{\paragraph{\sc #1}}\newcommand{\myfontscale}{\relscale{1.07}}
\UseGaramond{GaramondNo8}\newcommand{\myparagraph}[1]{\paragraph{#1}}\newcommand{\myfontscale}{\relscale{1.07}}
\usepackage[letterpaper,left=1.20in,right=1.20in,top=1.30in,bottom=1.30in]{geometry}
%\renewcommand{\baselinestretch}{1.66}\normalsize % doublespace
\renewcommand{\baselinestretch}{1.17}\normalsize % e.g. 12 on 14
%\renewcommand{\baselinestretch}{1.26}\normalsize % e.g. 13 on 15
\setlength{\parindent}{0in}  % Don't indent paragraphs
\setlength{\parskip}{2.5ex}    % Add space between paragraphs
\newcommand{\ie}{i.e.\@\xspace}
\newcommand{\eg}{e.g.\@\xspace}
\newcommand{\cf}{cf.\@\xspace}
\newcommand{\etc}{etc.\@\xspace}
\newcommand{\viz}{viz.\@\xspace}
\newcommand{\vs}{vs.\@\xspace}
\newcommand{\pr}{\mathsf{P}} % probability
%\newcommand{\expct}{\mbox{\boldmath $\mathsf E\,$}}
%\newcommand{\expct}{\mathsf E\,}
%\newcommand{\cov}{\mathop{\rm cov}\nolimits}
%\newcommand{\var}{\mathop{\rm var}\nolimits}
%\newcommand{\var}{\sigma^2}
\newcommand{\cmnt}[1]{}
\renewcommand{\cmnt}[1]{{\color{red}[#1]}}
\newcommand{\fn}[1]{\footnote{#1}}

\author{Marshall Abrams, University of Alabama at Birmingham, mabrams@uab.edu}

\begin{document}
\myfontscale % only works in document, more or less

Let $a$ be the activation, and $a'$ be the next activation.
Then, if the incoming communication is .05, the update function
is:
\[
x' = \min(1,x + .05(1 - x)) = \min(1,.95x + .05)
\]
(A positive increment can't push the activation below the
minimum, so we can ignore that possibility.)

When the incoming communiction is -.05, the update function is:
\[
x' = 
\max(-1,x - .05(x \;\,-\;\;\; -1)) = \max(-1,x - .05(x + 1)) = 
\]
%
\[
\max(-1,x - .05x - .05)) = \max(-1, .95x - .05)
\]
%

So the general update function is
\[
.95x \pm .05
\]
with the additional constraint that values outside of [-1,1] are
mapped back to the nearest extremum.

Note that although the original formula was intended to reflect
distance of the activation from the extremum in the direction of
push, we see here that the formula reduces to one in which:
\begin{itemize}\vspace{-2ex}
\item We simply reduce the activation by 5\%, whatever the
activation was\ \ldots
\item Then add or subtract .05 (no scaling here)
\item And then reduce the result back to the extremeum, if necessary.
\end{itemize}
%

(Note that the $.95x$ is not a real {\em decay\/} function, since if
a node receives no transmission, this reduction isn't applied.  A
real decay function acts on every tick, regardless of what happens
with incoming transmissions.)

%\bibliographystyle{myforthcoming} % the .bst file
%\bibliography{phil}
\end{document}

