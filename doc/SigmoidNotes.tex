\documentclass[11pt]{article}
\usepackage{natbib}
\usepackage{xspace}
\usepackage{graphicx}
\usepackage{url}
\usepackage{relsize} % provides \relscale, \larger, etc. to be used *in the document*
\usepackage{color}
\XeTeXdashbreakstate=1 % Tell XeLaTeX it's OK to break lines after em-dashes.
\newcommand{\SetFont}[1]{\setmainfont[Ligatures=TeX]{#1}} % preferred to "Mapping=tex-text"
\newcommand{\UseGaramond}[1]{%
	\usepackage[urw-garamond]{mathdesign}  % goes before fontenc, fontspec, etc.
	\usepackage[T1]{fontenc}
	\usepackage{fontspec,xltxtra,xunicode}
	\SetFont{#1}}
%\UseGaramond{EB Garamond}\newcommand{\myparagraph}[1]{\paragraph{\sc #1}}\newcommand{\myfontscale}{\relscale{1.00}}
\UseGaramond{GaramondNo8}\newcommand{\myparagraph}[1]{\paragraph{#1}}\newcommand{\myfontscale}{\relscale{1.00}}
\usepackage[letterpaper,left=1.25in,right=1.25in,top=0.90in,bottom=0.90in]{geometry}
%\renewcommand{\baselinestretch}{1.66}\normalsize % doublespace
\renewcommand{\baselinestretch}{1.17}\normalsize % e.g. 12 on 14
%\renewcommand{\baselinestretch}{1.26}\normalsize % e.g. 13 on 15
\setlength{\parindent}{0in}  % Don't indent paragraphs
\setlength{\parskip}{2.5ex}    % Add space between paragraphs
\newcommand{\ie}{i.e.\@\xspace}
\newcommand{\eg}{e.g.\@\xspace}
\newcommand{\cf}{cf.\@\xspace}
\newcommand{\etc}{etc.\@\xspace}
\newcommand{\viz}{viz.\@\xspace}
\newcommand{\vs}{vs.\@\xspace}
\newcommand{\pr}{\mathsf{P}} % probability
%\newcommand{\expct}{\mbox{\boldmath $\mathsf E\,$}}
%\newcommand{\expct}{\mathsf E\,}
%\newcommand{\cov}{\mathop{\rm cov}\nolimits}
%\newcommand{\var}{\mathop{\rm var}\nolimits}
%\newcommand{\var}{\sigma^2}
\newcommand{\cmnt}[1]{}
\renewcommand{\cmnt}[1]{{\color{red}[#1]}}
\newcommand{\fn}[1]{\footnote{#1}}

\author{Marshall Abrams, University of Alabama at Birmingham, mabrams@uab.edu}

\begin{document}
\pagestyle{empty}
\myfontscale % only works in document, more or less

{\Large Notes on sigmoid functions}\\
{\large Marshall Abrams, University of Alabama at Birmingham}

In response to my question, Belov at math.stackexchange.com\fn{
    \tiny \url{http://math.stackexchange.com/questions/367078/computationally-simple-sigmoid-with-specific-slopes-at-specific-points}}
suggested at that the following function 
\[
    \tanh\frac{kx}{(1-x^2)^{1/n}}
\]
with $k>0$, $n\geq 1$ would be such that:
\begin{itemize}
\item derivatives at -1 and 1 are zero
\item derivative is not zero between -1 and 1
\item derivative at 0 is 1, maybe can be controlled with a
    parameter
\item I can control the change in derivative near -1 and 1 within
    $(-1,1)$, so that the derivative doesn't depart from zero too
    quickly or too slowly. The curve is "symmetric" in the sense
    that flipping the curve on $[0,1]$ both vertically and
    horizontally produces the curve on $[0,1]$.
\end{itemize}
My file GeneralSigmoid.gcx can be used to experiment with this
function in OS X Grapher.

Notes on the parameters:
\begin{itemize}
\item Increasing $k$ (e.g. from 2) makes the curve steeper in the
    center.
\item Giving $k$ a value between 0 and 1 can make the curve very
    flat in the center.
\item The higher $n$ is, the more gently the curve slopes toward
    -1 and 1.  (This effect depends on the value of $k$.)
\item Small values of $n$ (\eg less than 1) can create "shelves"
    along which the curve appears completely flat near -1 and 1.
    (This violates my original requirements.)
\end{itemize}
For a nice paradigmatic sigmoid, use \eg $k=2$ or $k=3$ with
$n=2$.  For a function that is mostly flat and then rises quickly
to shelves near -1 and 1, try \eg $k=.01$, $n=.35$.  The latter
might be used to create a ``success" function that normally
drifts randomly, but creates a valued or rejected cultvar when
it reaches an extreme value.

\end{document}

